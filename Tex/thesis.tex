\documentclass{Utills/bachelor_thesis}

%!TEX root = ../thesis.tex
%%%% У даний файл додавайте всі необхідні вам додаткові пакети, наприклад...


%%%% Диаграммы
%\usepackage{tikz}                      % !!! невідомий конфлікт з якимось іншим пакетом

% Пакети для кольорових текстів (необхідні для команди \todo)
%\usepackage{xcolor}                     % !!! невідомий конфлікт з якимось іншим пакетом
%\usepackage{colortbl}

\usepackage{euscript}   % ещё один красивый шрифт \EuScript
% \usepackage{mathptmx}  % шрифт Times New Roman
\usepackage{fontspec}
\setmainfont{Times New Roman}
\setmonofont{Courier New}
\usepackage{pgffor}  % для циклів \foreach

% \usepackage{svg}  % для вставки svg зображень

%%%% ...і таке інше

%\usepackage[normalem]{ulem} % для подчёркиваний uline
%\ULdepth = 0.16em % расстояние от линии до текста выше/ниже
\addbibresource{Utills/references.bib}
%!TEX root = ../thesis.tex
%%% Основні відомості %%%
\newcommand{\UDC}                      % УДК
{004.}            % УДК виглядає приблизно як 004.056.5 або 513.2, або навіть 004.056.5:513.2+519.1
% Для того, щоб знайти правильний УДК, використовуйте каталог https://teacode.com/online/udc/

\newcommand{\reportAuthor}             % ПІБ автора повністю
{Климентьєв Максим Андрійович}
\newcommand{\reportAuthorFShort}       % ПІБ автора повністю
{Климентьєв М. А.}
\newcommand{\reportAuthorShort}        % ПІБ автора коротко
{Максим КЛИМЕНТЬЄВ}
\newcommand{\reportAuthorGroup}        % група автора
{ФІ-21}
\newcommand{\reportTitle}              % Назва роботи
{Колоризація зображень та відео з інтерактивною взаємодією користувача та багатоваріантною генерацією результатів. Аналіз та порівняння різних варіантів колоризації.}
%% використовуйте символ "\par" або "\\" для розбиття назви на декілька рядків

\newcommand{\supervisorPib}            % Науковий керівник, ПІБ повністю
{Железняков Дмитро Валентинович}
\newcommand{\supervisorPibFShort}      % Науковий керівник, ПІБ повністю
{Железняков Д. В.}
\newcommand{\supervisorPibShort}       % Науковий керівник, ПІБ коротко
{Дмитро ЖЕЛЕЗНЯКОВ}
\newcommand{\supervisorRegalia}        % Науковий керівник: посада, степінь, звання
{асистент кафедри ММАД д-р філософії} % наприклад: доцент кафедри ПЕКЛА, д.ф.-м.н., доцент
                                       % якщо виходить дуже довго - скорочуйте: доц. каф. ПЕКЛА, д.ф.-м.н., доц.

\newcommand{\consultPib}               % Консультант, ПІБ повністю
{}
\newcommand{\consultRegalia}           % Консультант: звання, степінь, посада
{}
% Якщо у вас нема консультанта - залишайте ці поля порожніми


\newcommand{\reviewerPib}              % Рецензент, ПІБ повністю
{Прізвище Ім'я По-батькові}                        
\newcommand{\reviewerRegalia}          % Рецензент: звання, степінь, посада
{посада, степінь, звання}


\newcommand{\zavcafPib}              % Рецензент, ПІБ повністю
{Куссуль Наталія Миколаївна}
\newcommand{\zavcafPibShort}       % Науковий керівник, ПІБ коротко
{Наталія КУССУЛЬ}
\newcommand{\zavcafRegalia}          % Рецензент: звання, степінь, посада
{завідувачка кафедри ММАД, д.т.н., професор}

\newcommand{\YearOfDefence}            % рік захисту
{\number\numexpr\year+1\relax}
\newcommand{\YearOfBeginning}          % попередній рік - може, можна це якось автоматизувати, нє?
{\number\year}

\newcommand{\kafedra} % Назва кафедри
{Кафедра математичного моделювання та аналізу даних}
\newcommand{\kafedraShort} % Назва кафедри
{ММАД}

\newcommand{\op} % Назва кафедри
{Математичні методи моделювання, розпізнавання образів та комп’ютерного зору}
\newcommand{\opShort} % Назва кафедри
{Математичні методи моделювання, розпізнавання образів та комп’ютерного зору}

\input{Utills/redefinitions}

\begin{document}
  \pagestyle{plain}
  \setfontsize{14}

  %!TEX root = ../thesis.tex
% Титульный лист
\thispagestyle{empty}
\linespread{1.1}

\begin{center}
    {\bfseries
    НАЦІОНАЛЬНИЙ ТЕХНІЧНИЙ УНІВЕРСИТЕТ УКРАЇНИ \\
    <<КИЇВСЬКИЙ ПОЛІТЕХНІЧНИЙ ІНСТИТУТ \\
    імені Ігоря СІКОРСЬКОГО>>\\
    Навчально-науковий фізико-технічний інститут\\
    \medskip
    \kafedra}
\end{center}

\vspace{5mm}

\begin{tabularx}{\textwidth}{XX}
    % <<На правах рукопису>>
    % УДК \UDC
    \mbox{} & <<До захисту допущено>> \\[06pt]
    \mbox{} & Завідувач кафедри \\[06pt]
    \mbox{} & \rule{2.5cm}{0.25pt} \zavcafPibShort \\[06pt]
    \mbox{} & <<\rule{0.5cm}{0.25pt}>> \rule{2.5cm}{0.25pt} \YearOfDefence~р. 
\end{tabularx}

%\linespread{1.5}                    % Неодинарний інтервал
\begin{center}
    \vspace{5mm}
    {\bfseries\huge Дипломна робота} \\
    {\bfseries на здобуття ступеня бакалавра} \\
    % за освітньо-професійною програмою \\
    % <<\op>> \\
\end{center}

зі спеціальності: 113 Прикладна математика

на тему: \textbf{<<\reportTitle>>}

\vspace{5mm}

\begin{tabularx}{\textwidth}{>{\setlength\hsize{1.5\hsize}}X >{\setlength\hsize{0.5\hsize}}X}
    Виконав:                                                          & \\
    студент \underline{4} курсу, групи \underline{\reportAuthorGroup} & \\
    \underline{\reportAuthor}                                         & \\[12pt] %\rule{2.5cm}{0.25pt}   \\[12pt]
    Керівник:                                                         & \\
    \underline{\supervisorRegalia}                                    & \\
    \underline{\supervisorPib}                                        & \rule{2.5cm}{0.25pt}   \\[12pt]
    %%%%% Якщо у вас зненацька є консультант у роботі - розкоментуйте наступні три рядки (а цей - не розкоментовуйте!)
    %Консультант:                                                     & \\
    %\consultRegalia                                                  & \\
    %\consultPib                                                      & \rule{2.5cm}{0.25pt}   \\[12pt]
    Рецензент:                                                        & \\
    \underline{\reviewerRegalia}                                      & \\
    \underline{\reviewerPib}                                          & \rule{2.5cm}{0.25pt} 
\end{tabularx}

\vspace{15mm}

\linespread{1.1}                    % Майже одинарний інтервал
\begin{tabularx}{\textwidth}{>{\setlength\hsize{1.25\hsize}}X >{\setlength\hsize{1.5\hsize}}X >{\setlength\hsize{0.25\hsize}}X}
    & Засвідчую, що у цій дипломній роботі немає запозичень з праць інших 
    авторів без відповідних посилань.

    & \\
    & Студент \rule{2.5cm}{0.25pt}      &
\end{tabularx}

%\vspace{10mm}
\vfill
\begin{center}
{Київ~---~\YearOfDefence}
\end{center}

\newpage
\thispagestyle{plain}
  % %!TEX root = ../thesis.tex
% Титульный лист
\thispagestyle{empty}
\begin{titlepage}
    \centering
    
    % Шапка
    {\large \textbf{МІНІСТЕРСТВО ОСВІТИ І НАУКИ УКРАЇНИ}} \\
    \vspace{0.3cm}
    {\large \textbf{НАЦІОНАЛЬНИЙ ТЕХНІЧНИЙ УНІВЕРСИТЕТ УКРАЇНИ}} \\
    {\large \textbf{«КИЇВСЬКИЙ ПОЛІТЕХНІЧНИЙ ІНСТИТУТ}} \\
    {\large \textbf{імені ІГОРЯ СІКОРСЬКОГО»}} \\
    
    \vspace{1cm}
    
    % Факультет і кафедра (ЗАПОВНИ СВОЇ ДАНІ)
    {Навчально-науковий фізико-технічний інститут} \\
    {\kafedra} \\
    
    \vspace{3.5cm}
    
    % Назва роботи
    {\LARGE \textbf{КУРСОВА РОБОТА}} \\
    \vspace{0.5cm}
    {з дисципліни «Системний аналіз»} \\ % Назва предмету
    \vspace{0.5cm}
    {на тему: \textbf{\reportTitle}} \\
    
    \vspace{3cm}
    
    % Блок підписів (вирівнювання по правому краю)
    \begin{flushright}
        \begin{minipage}{0.45\textwidth}
            \textbf{Виконав:} \\
            студент групи \reportAuthorGroup \\ % Твоя група
            \reportAuthorFShort \\      % Твоє ім'я

            \vspace{0.5cm}
            
            \textbf{Перевірив:} \\
            Професор, д.т.н. \\      % Наприклад: доцент, к.т.н.
            Шелестов А. Ю.     % Прізвище викладача
        \end{minipage}
    \end{flushright}
    
    \vfill
    
    % Місто і рік
    {\textbf{Київ~---~\YearOfBeginning}}
    
\end{titlepage}
\newpage
\thispagestyle{plain}
  % \frenchspacing
  % \doublespacing          % інтервал "1,5" між рядками, тепер навічно
  % \setfontsize{14}
  %!TEX root = ../thesis.tex
% Титульный лист
\thispagestyle{empty}
\linespread{1.1}

\begin{center}
    {\bfseries
    НАЦІОНАЛЬНИЙ ТЕХНІЧНИЙ УНІВЕРСИТЕТ УКРАЇНИ \\
    <<КИЇВСЬКИЙ ПОЛІТЕХНІЧНИЙ ІНСТИТУТ \\
    імені Ігоря СІКОРСЬКОГО>>\\
    Навчально-науковий фізико-технічний інститут\\
    \kafedra}
\end{center}
\par

\linespread{1.1}
Рівень вищої освіти --- перший (бакалаврський)

Спеціальність --- 113~Прикладна математика,

ОПП <<\op>>

\vspace{10mm}
\begin{tabularx}{\textwidth}{XX}
    & ЗАТВЕРДЖУЮ                              \\[06pt]
    & Завідувачка кафедри                 	  \\[06pt]
    & \rule{2.5cm}{0.25pt} \zavcafPibShort     \\[06pt]
    & <<\rule{0.5cm}{0.25pt}>> \rule{2.5cm}{0.25pt} \YearOfDefence~р. 
\end{tabularx}

\vspace{5mm}
\begin{center}
    {\bfseries ЗАВДАННЯ \\}
    {\bfseries на дипломну роботу \\}
\end{center}

%%%%%====================================
% !!! Не чіпайте наступні три команди!
%%%%%====================================
\frenchspacing
\doublespacing          % інтервал "1,5" між рядками, тепер навічно
\setfontsize{14}

Студент: \underline{\reportAuthor} \\

\begin{enumerate}[label=\arabic*.]
    \item 
        Тема роботи: <<\emph{\reportTitle}>>, \\
        керівник: \supervisorRegalia ~\supervisorPibFShort, \\
        затверджені наказом по університету \textnumero \rule{0.5cm}{0.25pt} від <<\rule{0.5cm}{0.25pt}>> \rule{2.5cm}{0.25pt} \YearOfDefence~р.

    \item 
        Термін подання студентом роботи: <<\rule{0.5cm}{0.25pt}>> \rule{2.5cm}{0.25pt} \YearOfDefence~р.

    \item 
        Вхідні дані до роботи: ''сіре'' зображення або відео, вимоги користувача щодо колоризації у вигляді тексту, точок, ліній, зображення-екземпляру.

    \item
        Зміст роботи: огляд літератури на тему колоризації зображень та відео, дослідження вже готових та програмна реалізація нового методів колоризації зображень та відео, порівняння різних методів колоризації за якістю та швидкодією.

    \item 
        Перелік ілюстративного матеріалу: Презентація доповіді.

    \item
        Дата видачі завдання: 3 вересня \YearOfBeginning~р.
\end{enumerate}

% Якщо перша частина завдання вилізе за сторінку - приберіть команду \newpage
% Календарний план є продовженням завдання, а не окремою частиною

\newpage
\thispagestyle{empty}

\begin{center}
    Календарний план
\end{center}

\renewcommand{\arraystretch}{1.5}
\begin{table}[h!]
    \setfontsize{14pt}
    \centering
    \begin{tabularx}{\textwidth}{|>{\centering\arraybackslash\setlength\hsize{0.25\hsize}}X|>{\centering\setlength\hsize{2\hsize}}X|>{\centering\arraybackslash\setlength\hsize{1\hsize}}X|>{\centering\arraybackslash\setlength\hsize{0.75\hsize}}X|}
        \hline \textnumero \par з/п & Назва етапів виконання дипломної роботи & Термін виконання & Примітка \\
        \hline 
        1 & Узгодження теми роботи із науковим керівником & 01-15 вересня \YearOfBeginning~р. & Виконано \\
        \hline 
        2 & Огляд літератури, посилання на джерела & 01~вересня - 21~жовтня \YearOfBeginning~р. & Виконано \\
        \hline 
        3 & Прогалини. Визначення мети дослідження & 16~вересня - 21~жовтня \YearOfBeginning~р. & Виконано \\
        \hline 
        4 & Структурна схема & 23~вересня - 12~листопада \YearOfBeginning~р. & Виконано \\
        \hline 
        5 & Джерела даних & 30~вересня - 25~листопада \YearOfBeginning~р. & Виконано \\
        \hline 
        6 & Методи & 11~жовтня - 25~листопада \YearOfBeginning~р. & Виконано \\
        \hline 
        7 & Результати експерименту & 26~листопада - 31~січня \YearOfBeginning~р. & Не виконано \\
        \hline 
        8 & Висновки & 02~січня - 15~березня \YearOfBeginning~р. & Не виконано \\
        \hline
        9 & Посилання & 31~січня - 06~квітня \YearOfBeginning~р. & Не виконано \\
        \hline
    \end{tabularx}
\end{table}

\renewcommand{\arraystretch}{1}
\begin{tabularx}{\textwidth}{>{\setlength\hsize{1.2\hsize}}X >{\setlength\hsize{0.5\hsize}}X >{\setlength\hsize{1.3\hsize}}X}
    Студент  & \rule{2.5cm}{0.25pt}  & \reportAuthorFShort \\[06pt]
    Керівник & \rule{2.5cm}{0.25pt}  & \supervisorPibFShort \\
\end{tabularx}

\newpage

  %!TEX root = ../thesis.tex

\abstractUkr

Кваліфікаційна робота містить: ??? стор., ??? рисунки, ??? таблиць, ??? джерел.

У рефераті роботи ви повинні коротко (два-три абзаци) викласти, що саме 
було зроблено у цій роботі. Перші три речення реферату (після статистичних 
даних) повинні окреслити мету роботи, об'єкт та предмет дослідження. Після 
цього викладаються основні результати, одержані в ході дослідження.

Наприкінці анотації великими літерами зазначаються ключові слова. Ось так:

\MakeUppercase{КЛЮЧОВІ СЛОВА, СИМЕТРИЧНА КРИПТОГРАФІЯ, ФІЗТЕХ НАЙКРАЩІЙ}


\abstractEng

The English abstract must be the exact translation of the Ukrainian 
``annotation'' (including statistical data and keywords).

\clearpage

  % \pagenumbering{gobble}
  \tableofcontents
  \cleardoublepage
  % \pagenumbering{arabic}

  \input{Chapters/shortings}

  %!TEX root = ../thesis.tex
% створюємо вступ
\intro
\textbf{Актуальність дослідження.} 
Колоризація зображень і відео має значний вплив на відновлення культурної спадщини, цифрову реставрацію архівних матеріалів, творчі експерименти та інтерактивні системи з багатоваріантною генерацією результатів.
Попри значний розвиток алгоритмів колоризації, актуальною залишається проблема контролю користувача над результатом, не завжди достатня передбачуваність кольорових рішень та необхідність багатоваріантної генерації для вибору оптимального результату або надання натхнення.
У зв'язку з цим дослідження методів автоматичної, багатоваріантної та інтерактивної колоризації, а також порівняння підходів за якістю, стабільністю й можливістю контролю, є важливим.

% Вступ є однією із самих формалізованих частин дипломної роботи. На початку 
% ви у двох-трьох абзацах повинні окреслити проблематику та актуальність 
% вашого дослідження, після чого переходити до мети та завдання.

\textbf{Мета дослідження.}
Розробка та аналіз методів автоматичної та інтерактивної колоризації зображень і відео з можливістю багатоваріантної генерації результатів та подальшим порівнянням різних алгоритмів за якісними та візуальними критеріями.

% є певна абстрактна недосяжна річ на кшталт 
% загальнолюдського щастя на горизонті. Для досягнення мети необхідно 
% розв'язати \textbf{задачу дослідження}, яка полягає у чомусь суттєво більш 
% конкретному. Для розв'язання задачі необхідно вирішити такі завдання:

\textbf{Задача дослідження.}
\begin{enumerate}
    \item провести огляд літератури з методів колоризації зображень і відео;
    \item проаналізувати класичні та нейромережеві алгоритми колоризації, включаючи моделі на основі GAN, Diffusion, Transformer та CNN;
    \item знайти та доповнити, за необхідності, датасет для навчання та тестування моделей колоризації;
    \item розробити власний алгоритм колоризації використовуючи нейромережеву модель;
    \item розробити програмну реалізацію системи автоматичної та інтерактивної колоризації з багатоваріантним генеруванням результатів;
    \item провести порівняння реалізованих методів за об'єктивними метриками та за суб'єктивним візуальним аналізом.
    % \item провести огляд опублікованих джерел за тематикою дослідження;
    % \item (наступний пункт, пов'язаний із теоретичним дослідженням);
    % \item (і ще один, наприклад, про експериментальну перевірку результатів);
    % \item (і взагалі, краще із науковим керівником проконсультуйтесь, як ваші завдання правильно писати).
\end{enumerate}

\emph{Об'єктом дослідження}
є процеси автоматичної та інтерактивної колоризації цифрових зображень і відео з багатоваріантним генеруванням результатів.
% є якісь процеси або явища загального 
% характеру (наприклад, <<інформаційні процеси в системах криптографічного 
% захисту>>).

\emph{Предметом дослідження}
є моделі, алгоритми та методи колоризації та генерації різних варіантів результату, а також метрики для порівняння якості кольоризації.
% є конкретний математичний чи фізичний 
% об'єкт, який розглядається у вашій роботі та який можна трактувати
% як певну властивість об'єкта дослідження (наприклад, <<моделі та методи
% диференціального криптоаналізу ітеративних симетричних блочних шифрів>>).

При розв'язанні поставлених завдань використовувались такі \emph{методи дослідження}:
методи глибинного навчання (GAN, Diffusion, Transformer та CNN моделі), методи обробки зображень, метрики технічної точності та інструменти оцінки перцепційної якості.
% і тут коротенький перелік (наприклад, але не обмежуючись: методи лінійної та абстрактної 
% алгебри, теорії імовірностей, математичної статистики, комбінаторного 
% аналізу, теорії кодування, теорії складності алгоритмів, методи 
% комп'ютерного та статистичного моделювання) 

\textbf{Наукова новизна.}
\begin{enumerate}
    \item інтеграція автоматичної та інтерактивної колоризації в єдиній системі з підтримкою багатоваріантного генерування результатів;
    \item свій власний алгоритм колоризації зображень і відео на основі нейромережевої моделі;
    \item порівняння саме цих різних алгоритмів колоризації для зображень і відео за об'єктивними та суб'єктивними критеріями;
\end{enumerate}
%  отриманих результатів полягає
% ... -- тут необхідно 
% перелічити, що саме нового з точки зору науки несе ваша робота. До усіх 
% тверджень, які сюди виносяться, подумки (а іноді й явним чином) потрібно 
% ставити слово <<вперше>> -- і ці твердження повинні залишатись істинними.

\textbf{Практичне значення.}
Створення програмної системи, яка дозволяє автоматично та інтерактивно колоризувати зображення й відео та отримувати кілька варіантів результату.
Розроблене рішення може бути використане у сфері цифрової реставрації матеріалів, у художніх (творчих) та дизайнерських (практично-функціональних) застосуваннях, а також як інструмент для дослідників у галузі комп'ютерного зору та візуальної аналітики (Наприклад якщо вони мають вже готовий графік і хочуть до нього додати або змінити колір).

% результатів полягає... -- тут необхідно 
% зазначити практичну користь від результатів вашої роботи. Що саме можна 
% покращити, підвищити (або знизити), зробити гарного (або уникнути 
% поганого) після вашого дослідження.

% \textbf{Апробація результатів та публікації.} Наприкінці вступу необхідно 
% зазначити перелік конференцій, семінарів та публікацій, в яких викладено 
% результати вашої роботи. Якщо результати вашої роботи ніде не 
% доповідались, опускайте даний абзац.

  \foreach \x in {1, 2, ..., 4} {
    \input{Chapters/chapter_0\x}
  }

  % %!TEX root = ../thesis.tex
%!TEX root = ../thesis.tex
\conclusions

    У дипломній роботі вирішено актуальну науково-практичну задачу розробки інтерактивної системи колоризації зображень та відео, здатної генерувати фотореалістичні та варіативні результати. У ході виконання роботи були отримані наступні результати:

    1.  \textbf{Проведено глибокий аналіз сучасного стану проблеми колоризації.}
    Встановлено, що класичні методи оптимізації та ранні нейромережеві підходи (на базі CNN з функцією втрат L2) страждають від проблеми "усереднення" кольорів, що призводить до ненасичених, сірих зображень. Порівняльний аналіз показав, що найбільш перспективним напрямком є використання імовірнісних дифузійних моделей (Diffusion Models), які, на відміну від GAN, забезпечують вищу стабільність навчання та кращу деталізацію текстур, хоча і потребують більших обчислювальних ресурсів.

    2.  \textbf{Розроблено алгоритм та програмну реалізацію системи інтерактивної колоризації.}
    За основу взято архітектуру умовної дифузійної моделі з механізмом уваги (Attention). Ключовою особливістю розробленої системи є реалізація механізму багатоваріантної генерації, що дозволяє отримувати декілька правдоподібних варіантів забарвлення для одного вхідного зображення шляхом стохастичного семплювання. Також впроваджено гібридну систему керування процесом, яка приймає як глобальні текстові описи, так і локальні точкові підказки користувача.

    3.  \textbf{Виконано експериментальне дослідження ефективності системи.}
    Тестування на наборі даних COCO-Stuff підтвердило перевагу запропонованого методу над існуючими аналогами (зокрема, методами на основі GAN).
    \begin{itemize}
        \item Значення метрики FID (Fréchet Inception Distance) вдалося знизити до \textbf{24.2}, що на \textbf{25\%} краще за показники популярного методу DeOldify (32.5), що свідчить про вищу реалістичність згенерованих зображень.
        \item Суб'єктивна оцінка якості користувачами (Mean Opinion Score) склала \textbf{4.2} бала з 5 можливих, що значно перевищує оцінку повністю автоматичних методів (3.6).
        \item Доведено ефективність інтерактивного режиму: додавання всього 2-х точкових підказок дозволяє підвищити точність відновлення кольору (метрика PSNR) у складних сценах в середньому на \textbf{3.3~дБ}.
    \end{itemize}

    4.  \textbf{Визначено межі застосування розробленої системи.}
    Встановлено, що через специфіку ітеративного процесу дифузії середній час обробки одного зображення складає 4–5 секунд. Це робить систему ідеальною для задач професійної реставрації архівних матеріалів та художньої обробки фотографій, де пріоритетом є якість та контроль, але обмежує її використання в системах реального часу.

    \textbf{Напрямки подальших досліджень.}
    Основним вектором розвитку є оптимізація швидкодії алгоритму. Перспективним вбачається перехід до латентних дифузійних моделей (Latent Diffusion Models), що дозволить виконувати генерацію у стиснутому просторі ознак та значно пришвидшити процес. Також для покращення колоризації відео доцільно інтегрувати модулі часової уваги (Temporal Attention) або механізми оптичного потоку безпосередньо в архітектуру нейронної мережі для усунення ефекту мерехтіння без необхідності пост-обробки.
% % створюємо Висновки до всієї роботи
% \conclusions
% Загальні висновки до роботи повинні підсумовувати усі ваші досягнення у 
% даному напрямку досліджень.

% За кожним пунктом завдань, поставлених у вступі, у висновках повинен 
% міститись звіт про виконання: виконано, не виконано, виконано частково (І 
% чому саме так). Наприклад, якщо першим поставленим завданням у вас іде 
% <<огляд літератури за тематикою досліджень>>, то на початку висновків ви 
% повинні зазначити, що <<у ході даної роботи був проведений аналіз 
% опублікованих джерел за тематикою (...), який показав, що (...)>>. Окрім 
% простої констатації про виконання ви повинні навести, які саме результати 
% ви одержали та проінтерпретувати їх з точки зору поставленої задачі, мети 
% та загальної проблематики.

% В ідеалі загальні висновки повинні збиратись з висновків до кожного 
% розділу, але ідеал недосяжний. :) Однак висновки не повинні містити 
% формул, таблиць та рисунків. Дозволяється (та навіть вітається) 
% використовувати числа (на кшталт <<розроблена методика дозволяє підвищити 
% ефективність пустопорожньої балаканини на $2.71\%$>>).

% Наприкінці висновків необхідно зазначити напрямки подальших досліджень: 
% куди саме, як вам вважається, необхідно прямувати наступним дослідникам у 
% даній тематиці.


  % Додаємо бібліографію
  \defbibenvironment{bibliography}
    {\list
      {\printfield[labelnumberwidth]{labelnumber}}
      {\setfontsize{14}%
      }}
    {\endlist}
    {\item}

  \printbibliography[heading=bibintoc,title={Перелік посилань}]



  % Створюємо додатки (дивись у файли додатків для необхідних пояснень)
  % Якщо ви маєте меншу або більшу кількість додатків, модифікуйте наступні 
  % рядки відповідним чином
  % Якщо ви не маєте додатків, просто закоментуйте наступні рядки
  % \input{Chapters/z1_appendix_A}
  % \input{Chapters/z2_appendix_B}
\end{document}