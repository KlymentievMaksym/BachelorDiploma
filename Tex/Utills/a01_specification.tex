% Титульный лист
\linespread{1.1}

\begin{center}
    {\bfseries
    НАЦІОНАЛЬНИЙ ТЕХНІЧНИЙ УНІВЕРСИТЕТ УКРАЇНИ \\
    <<КИЇВСЬКИЙ ПОЛІТЕХНІЧНИЙ ІНСТИТУТ \\
    імені Ігоря СІКОРСЬКОГО>>\\
    Навчально-науковий фізико-технічний інститут\\
    \kafedra}
\end{center}
\par

\linespread{1.1}
Рівень вищої освіти --- перший (бакалаврський)

Спеціальність --- 113~Прикладна математика,

ОПП <<\op>>

\vspace{10mm}
\begin{tabularx}{\textwidth}{XX}
    & ЗАТВЕРДЖУЮ                              \\[06pt]
    & Завідувач кафедри                 	  \\[06pt]
    & \rule{2.5cm}{0.25pt} \zavcafFioShort     \\[06pt]
    & <<\rule{0.5cm}{0.25pt}>> \rule{2.5cm}{0.25pt} \YearOfDefence~р. 
\end{tabularx}

\vspace{5mm}
\begin{center}
    {\bfseries ЗАВДАННЯ \\}
    {\bfseries на дипломну роботу \\}
\end{center}

%%%%%====================================
% !!! Не чіпайте наступні три команди!
%%%%%====================================
\frenchspacing
\doublespacing          % інтервал "1,5" між рядками, тепер навічно
\setfontsize{14}

Студент: \reportAuthor \\

\begin{enumerate}
    \item 
        Тема роботи: <<\emph{\reportTitle}>>,
        науковий керівник дисертації: \supervisorRegalia ~\supervisorFio, \\
        затверджені наказом по університету \No \rule{0.5cm}{0.25pt} від <<\rule{0.5cm}{0.25pt}>> \rule{2.5cm}{0.25pt} \YearOfDefence~р.

    \item 
        Термін подання студентом роботи: <<\rule{0.5cm}{0.25pt}>> \rule{2.5cm}{0.25pt} \YearOfDefence~р.

    \item 
        % Коли будете заповнювати пункти 3-10, приберіть команду \emph --- вона тільки для виділення моїх коментарів
        Об'єкт дослідження: \emph{(впишіть об'єкт дослідження)}

    \item 
        Предмет дослідження: \emph{(впишіть предмет дослідження)}

    \item
        Перелік завдань: \emph{(впишіть теми та задачі, які ви розкриваєте у роботі; можна робити це попунктно)}

    \item
        Орієнтовний перелік графічного (ілюстративного) матеріалу: \emph{(якщо у вас є окремий ілюстративний матеріал окрім власне роботи (креслення, макети тощо), зазначайте; інакше вказуйте <<Презентація доповіді>>)}

    \item
        Орієнтовний перелік публікацій: \emph{(впишіть наявні публікації або <<планується доповідь на всеукраїнській конференції>>)}

    \item
        Дата видачі завдання: 3 вересня \YearOfBeginning~р.
\end{enumerate}

% Якщо перша частина завдання вилізе за сторінку - приберіть команду \newpage
% Календарний план є продовженням завдання, а не окремою частиною

% \newpage

\begin{center}
    Календарний план
\end{center}

\renewcommand{\arraystretch}{1.5}
\begin{table}[h!]
    \setfontsize{14pt}
    \centering
    \begin{tabularx}{\textwidth}{|>{\centering\arraybackslash\setlength\hsize{0.25\hsize}}X|>{\setlength\hsize{2\hsize}}X|>{\centering\arraybackslash\setlength\hsize{1\hsize}}X|>{\centering\arraybackslash\setlength\hsize{0.75\hsize}}X|}
        \hline \No\par з/п & Назва етапів виконання ??магістерської дисертації?? & Термін виконання & Примітка \\
        \hline 
        % номер етапу
        1 & 
        % назва етапу
        Узгодження теми роботи із науковим керівником & 
        % термін виконання
        01-15 вересня \YearOfBeginning~р. &
        % примітка - зазвичай "Виконано"
        Виконано \\
    %%% -- початок інтервалу для копіювання
        \hline 
        % номер етапу
        2 & 
        % назва етапу
        Огляд опублікованих джерел за тематикою дослідження & 
        % термін виконання
        Вересень-жовтень \YearOfBeginning~р. &
        % примітка - зазвичай "Виконано"
        Виконано \\
    %%% -- кінець інтервалу для копіювання
    % не прибирайте амперсанди та \\ наприкінці рядків!
    % скопійовані інтервали вставляти перед фінальною \hline та заповнювати відповідно
    % ось так:
    %%% -- початок інтервалу для копіювання
        \hline 
        % номер етапу
        3 & 
        % назва етапу
        \ldots & 
        % термін виконання
        \ldots &
        % примітка - зазвичай "Виконано"
        Виконано \\
    %%% -- кінець інтервалу для копіювання
        \hline %фінальна hline
    \end{tabularx}
\end{table}

\renewcommand{\arraystretch}{1}
\begin{tabularx}{\textwidth}{>{\setlength\hsize{1.2\hsize}}X >{\setlength\hsize{0.5\hsize}}X >{\setlength\hsize{1.3\hsize}}X}
    Студент  & \rule{2.5cm}{0.25pt}  & \reportAuthorShort \\[06pt]
    Керівник & \rule{2.5cm}{0.25pt}  & \supervisorFioShort \\
\end{tabularx}

\newpage
