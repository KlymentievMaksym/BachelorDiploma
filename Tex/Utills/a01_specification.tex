%!TEX root = ../thesis.tex
% Титульный лист
\linespread{1.1}

\begin{center}
    {\bfseries
    НАЦІОНАЛЬНИЙ ТЕХНІЧНИЙ УНІВЕРСИТЕТ УКРАЇНИ \\
    <<КИЇВСЬКИЙ ПОЛІТЕХНІЧНИЙ ІНСТИТУТ \\
    імені Ігоря СІКОРСЬКОГО>>\\
    Навчально-науковий фізико-технічний інститут\\
    \kafedra}
\end{center}
\par

\linespread{1.1}
Рівень вищої освіти --- перший (бакалаврський)

Спеціальність --- 113~Прикладна математика,

ОПП <<\op>>

\vspace{10mm}
\begin{tabularx}{\textwidth}{XX}
    & ЗАТВЕРДЖУЮ                              \\[06pt]
    & Завідувач кафедри                 	  \\[06pt]
    & \rule{2.5cm}{0.25pt} \zavcafFioShort     \\[06pt]
    & <<\rule{0.5cm}{0.25pt}>> \rule{2.5cm}{0.25pt} \YearOfDefence~р. 
\end{tabularx}

\vspace{5mm}
\begin{center}
    {\bfseries ЗАВДАННЯ \\}
    {\bfseries на дипломну роботу \\}
\end{center}

%%%%%====================================
% !!! Не чіпайте наступні три команди!
%%%%%====================================
\frenchspacing
\doublespacing          % інтервал "1,5" між рядками, тепер навічно
\setfontsize{14}

Студент: \reportAuthor \\

\begin{enumerate}
    \item 
        Тема роботи: <<\emph{\reportTitle}>>,
        науковий керівник дисертації: \supervisorRegalia ~\supervisorFio, \\
        затверджені наказом по університету \textnumero \rule{0.5cm}{0.25pt} від <<\rule{0.5cm}{0.25pt}>> \rule{2.5cm}{0.25pt} \YearOfDefence~р.

    \item 
        Термін подання студентом роботи: <<\rule{0.5cm}{0.25pt}>> \rule{2.5cm}{0.25pt} \YearOfDefence~р.

    \item 
        % Коли будете заповнювати пункти 3-10, приберіть команду \emph --- вона тільки для виділення моїх коментарів
        Об'єкт дослідження: ???процес автоматичної колоризації зображень та відео за допомогою алгоритмів штучного інтелекту???

    \item 
        Предмет дослідження: ???методи та алгоритми інтерактивної колоризації зображень та відео, включно з багатоваріантною генерацією результатів та їх порівнянням???

    \item
        Перелік завдань:
        \begin{itemize}
            \item Аналіз сучасних методів колоризації зображень та відео;
            \item Розробка алгоритму інтерактивної колоризації з можливістю багатоваріантного вибору результату;
            \item Реалізація прототипу системи колоризації;
            \item Проведення експериментального порівняння різних методів та оцінка їх якості.
            % \item Підготовка рекомендацій щодо оптимальних підходів у колоризації медіаконтенту
        \end{itemize}

    \item
        Орієнтовний перелік графічного (ілюстративного) матеріалу: \emph{(якщо у вас є окремий ілюстративний матеріал окрім власне роботи (креслення, макети тощо), зазначайте; інакше вказуйте <<Презентація доповіді>>)}  ???Презентація доповіді, зразки колоризованих зображень та відео, схеми алгоритмів та інтерфейсу системи???

    \item
        Орієнтовний перелік публікацій: \emph{(впишіть наявні публікації або <<планується доповідь на всеукраїнській конференції>>)} ???планується доповідь на всеукраїнській конференції???

    \item
        Дата видачі завдання: 3 вересня \YearOfBeginning~р.
\end{enumerate}

% Якщо перша частина завдання вилізе за сторінку - приберіть команду \newpage
% Календарний план є продовженням завдання, а не окремою частиною

% \newpage

\begin{center}
    Календарний план
\end{center}

\renewcommand{\arraystretch}{1.5}
\begin{table}[h!]
    \setfontsize{14pt}
    \centering
    \begin{tabularx}{\textwidth}{|>{\centering\arraybackslash\setlength\hsize{0.25\hsize}}X|>{\setlength\hsize{2\hsize}}X|>{\centering\arraybackslash\setlength\hsize{1\hsize}}X|>{\centering\arraybackslash\setlength\hsize{0.75\hsize}}X|}
        \hline \textnumero \par з/п & Назва етапів виконання ??магістерської дисертації?? & Термін виконання & Примітка \\
        \hline 
        1 & Узгодження теми роботи із науковим керівником & 01-15 вересня \YearOfBeginning~р. & Виконано \\
        \hline 
        2 & Огляд літератури, посилання на джерела & 01~вересня - 21~жовтня \YearOfBeginning~р. & Виконано \\
        \hline 
        3 & Прогалини. Визначення мети дослідження & 16~вересня - 21~жовтня \YearOfBeginning~р. & Виконано \\
        \hline 
        4 & Структурна схема & 23~вересня - 12~листопада \YearOfBeginning~р. & Виконано \\
        \hline 
        5 & Джерела даних & 30~вересня - 25~листопада \YearOfBeginning~р. & Виконується \\
        \hline 
        6 & Методи & 11~жовтня - 25~листопада \YearOfBeginning~р. & Виконується \\
        \hline 
        7 & Результати експерименту & 26~листопада - 31~січня \YearOfBeginning~р. & Не виконано \\
        \hline 
        8 & Висновки & 02~січня - 15~березня \YearOfBeginning~р. & Не виконано \\
        \hline
        9 & Посилання & 31~січня - 06~квітня \YearOfBeginning~р. & Не виконано \\
        \hline
    \end{tabularx}
\end{table}

\renewcommand{\arraystretch}{1}
\begin{tabularx}{\textwidth}{>{\setlength\hsize{1.2\hsize}}X >{\setlength\hsize{0.5\hsize}}X >{\setlength\hsize{1.3\hsize}}X}
    Студент  & \rule{2.5cm}{0.25pt}  & \reportAuthorShort \\[06pt]
    Керівник & \rule{2.5cm}{0.25pt}  & \supervisorFioShort \\
\end{tabularx}

\newpage
