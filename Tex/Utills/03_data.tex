%%% Основні відомості %%%
\newcommand{\UDC}                      % УДК
{004.}            % УДК виглядає приблизно як 004.056.5 або 513.2, або навіть 004.056.5:513.2+519.1
% Для того, щоб знайти правильний УДК, використовуйте каталог https://teacode.com/online/udc/

\newcommand{\reportAuthor}             % ПІБ автора повністю
{Климентьєв Максим Андрійович}
\newcommand{\reportAuthorShort}        % ПІБ автора коротко
{Максим КЛИМЕНТЬЄВ}
\newcommand{\reportAuthorGroup}        % група автора
{ФІ-21}
\newcommand{\reportTitle}              % Назва роботи
{Колоризація зображень та відео з інтерактивною взаємодією користувача та багатоваріантною генерацією результатів. Аналіз та порівняння різних варіантів колоризації.}
%% використовуйте символ "\par" або "\\" для розбиття назви на декілька рядків

\newcommand{\supervisorFio}            % Науковий керівник, ПІБ повністю
{Железняков Дмитро Валентинович}
\newcommand{\supervisorFioShort}       % Науковий керівник, ПІБ коротко
{Дмитро ЖЕЛЕЗНЯКОВ}
\newcommand{\supervisorRegalia}        % Науковий керівник: посада, степінь, звання
{Асистент кафедри ММАД, степінь, асистент} % наприклад: доцент кафедри ПЕКЛА, д.ф.-м.н., доцент
                                       % якщо виходить дуже довго - скорочуйте: доц. каф. ПЕКЛА, д.ф.-м.н., доц.

\newcommand{\consultFio}               % Консультант, ПІБ повністю
{}
\newcommand{\consultRegalia}           % Консультант: звання, степінь, посада
{}
% Якщо у вас нема консультанта - залишайте ці поля порожніми


\newcommand{\reviewerFio}              % Рецензент, ПІБ повністю
{Прізвище Ім'я По-батькові}                        
\newcommand{\reviewerRegalia}          % Рецензент: звання, степінь, посада
{посада, степінь, звання}


\newcommand{\zavcafFio}              % Рецензент, ПІБ повністю
{Куссуль Наталія Миколаївна}
\newcommand{\zavcafFioShort}       % Науковий керівник, ПІБ коротко
{Наталія КУССУЛЬ}
\newcommand{\zavcafRegalia}          % Рецензент: звання, степінь, посада
{Завідувачка кафедри ММАД, д.т.н., професор}

\newcommand{\YearOfDefence}            % рік захисту
{\number\numexpr\year+1\relax}
\newcommand{\YearOfBeginning}          % попередній рік - може, можна це якось автоматизувати, нє?
{\number\year}

\newcommand{\kafedra} % Назва кафедри
{Кафедра математичного моделювання та аналізу даних}
\newcommand{\kafedraShort} % Назва кафедри
{ММАД}

\newcommand{\op} % Назва кафедри
{Математичні методи моделювання, розпізнавання образів та комп’ютерного зору}
\newcommand{\opShort} % Назва кафедри
{Математичні методи моделювання, розпізнавання образів та комп’ютерного зору}
