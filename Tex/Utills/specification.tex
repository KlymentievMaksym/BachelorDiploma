%!TEX root = ../thesis.tex
% Титульный лист
\thispagestyle{empty}
\linespread{1.1}

\begin{center}
    {\bfseries
    НАЦІОНАЛЬНИЙ ТЕХНІЧНИЙ УНІВЕРСИТЕТ УКРАЇНИ \\
    <<КИЇВСЬКИЙ ПОЛІТЕХНІЧНИЙ ІНСТИТУТ \\
    імені Ігоря СІКОРСЬКОГО>>\\
    Навчально-науковий фізико-технічний інститут\\
    \kafedra}
\end{center}
\par

\linespread{1.1}
Рівень вищої освіти --- перший (бакалаврський)

Спеціальність --- 113~Прикладна математика,

ОПП <<\op>>

\vspace{10mm}
\begin{tabularx}{\textwidth}{XX}
    & ЗАТВЕРДЖУЮ                              \\[06pt]
    & Завідувачка кафедри                 	  \\[06pt]
    & \rule{2.5cm}{0.25pt} \zavcafPibShort     \\[06pt]
    & <<\rule{0.5cm}{0.25pt}>> \rule{2.5cm}{0.25pt} \YearOfDefence~р. 
\end{tabularx}

\vspace{5mm}
\begin{center}
    {\bfseries ЗАВДАННЯ \\}
    {\bfseries на дипломну роботу \\}
\end{center}

%%%%%====================================
% !!! Не чіпайте наступні три команди!
%%%%%====================================
\frenchspacing
\doublespacing          % інтервал "1,5" між рядками, тепер навічно
\setfontsize{14}

Студент: \underline{\reportAuthor} \\

\begin{enumerate}[label=\arabic*.]
    \item 
        Тема роботи: <<\emph{\reportTitle}>>, \\
        керівник: \supervisorRegalia ~\supervisorPibFShort, \\
        затверджені наказом по університету \textnumero \rule{0.5cm}{0.25pt} від <<\rule{0.5cm}{0.25pt}>> \rule{2.5cm}{0.25pt} \YearOfDefence~р.

    \item 
        Термін подання студентом роботи: <<\rule{0.5cm}{0.25pt}>> \rule{2.5cm}{0.25pt} \YearOfDefence~р.

    \item 
        Вхідні дані до роботи: ''сіре'' зображення або відео, вимоги користувача щодо колоризації у вигляді тексту, точок, ліній, зображення-екземпляру.

    \item
        Зміст роботи: огляд літератури на тему колоризації зображень та відео, дослідження вже готових та програмна реалізація нового методів колоризації зображень та відео, порівняння різних методів колоризації за якістю та швидкодією.

    \item 
        Перелік ілюстративного матеріалу: Презентація доповіді.

    \item
        Дата видачі завдання: 3 вересня \YearOfBeginning~р.
\end{enumerate}

% Якщо перша частина завдання вилізе за сторінку - приберіть команду \newpage
% Календарний план є продовженням завдання, а не окремою частиною

\newpage
\thispagestyle{empty}

\begin{center}
    Календарний план
\end{center}

\renewcommand{\arraystretch}{1.5}
\begin{table}[h!]
    \setfontsize{14pt}
    \centering
    \begin{tabularx}{\textwidth}{|>{\centering\arraybackslash\setlength\hsize{0.25\hsize}}X|>{\centering\setlength\hsize{2\hsize}}X|>{\centering\arraybackslash\setlength\hsize{1\hsize}}X|>{\centering\arraybackslash\setlength\hsize{0.75\hsize}}X|}
        \hline \textnumero \par з/п & Назва етапів виконання дипломної роботи & Термін виконання & Примітка \\
        \hline 
        1 & Узгодження теми роботи із науковим керівником & 01-15 вересня \YearOfBeginning~р. & Виконано \\
        \hline 
        2 & Огляд літератури, посилання на джерела & 01~вересня - 21~жовтня \YearOfBeginning~р. & Виконано \\
        \hline 
        3 & Прогалини. Визначення мети дослідження & 16~вересня - 21~жовтня \YearOfBeginning~р. & Виконано \\
        \hline 
        4 & Структурна схема & 23~вересня - 12~листопада \YearOfBeginning~р. & Виконано \\
        \hline 
        5 & Джерела даних & 30~вересня - 25~листопада \YearOfBeginning~р. & Виконано \\
        \hline 
        6 & Методи & 11~жовтня - 25~листопада \YearOfBeginning~р. & Виконано \\
        \hline 
        7 & Результати експерименту & 26~листопада - 31~січня \YearOfBeginning~р. & Не виконано \\
        \hline 
        8 & Висновки & 02~січня - 15~березня \YearOfBeginning~р. & Не виконано \\
        \hline
        9 & Посилання & 31~січня - 06~квітня \YearOfBeginning~р. & Не виконано \\
        \hline
    \end{tabularx}
\end{table}

\renewcommand{\arraystretch}{1}
\begin{tabularx}{\textwidth}{>{\setlength\hsize{1.2\hsize}}X >{\setlength\hsize{0.5\hsize}}X >{\setlength\hsize{1.3\hsize}}X}
    Студент  & \rule{2.5cm}{0.25pt}  & \reportAuthorFShort \\[06pt]
    Керівник & \rule{2.5cm}{0.25pt}  & \supervisorPibFShort \\
\end{tabularx}

\newpage
