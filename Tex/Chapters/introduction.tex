%!TEX root = ../thesis.tex
% створюємо вступ
\intro
\textbf{Актуальність дослідження.} 
Колоризація зображень і відео має значний вплив на відновлення культурної спадщини, цифрову реставрацію архівних матеріалів, творчі експерименти та інтерактивні системи з багатоваріантною генерацією результатів.
Попри значний розвиток алгоритмів колоризації, актуальною залишається проблема контролю користувача над результатом, не завжди достатня передбачуваність кольорових рішень та необхідність багатоваріантної генерації для вибору оптимального результату або надання натхнення.
У зв'язку з цим дослідження методів автоматичної, багатоваріантної та інтерактивної колоризації, а також порівняння підходів за якістю, стабільністю й можливістю контролю, є важливим.

% Вступ є однією із самих формалізованих частин дипломної роботи. На початку 
% ви у двох-трьох абзацах повинні окреслити проблематику та актуальність 
% вашого дослідження, після чого переходити до мети та завдання.

\textbf{Мета дослідження.}
Розробка та аналіз методів автоматичної та інтерактивної колоризації зображень і відео з можливістю багатоваріантної генерації результатів та подальшим порівнянням різних алгоритмів за якісними та візуальними критеріями.

% є певна абстрактна недосяжна річ на кшталт 
% загальнолюдського щастя на горизонті. Для досягнення мети необхідно 
% розв'язати \textbf{задачу дослідження}, яка полягає у чомусь суттєво більш 
% конкретному. Для розв'язання задачі необхідно вирішити такі завдання:

\textbf{Задача дослідження.}
\begin{enumerate}
    \item провести огляд літератури з методів колоризації зображень і відео;
    \item проаналізувати класичні та нейромережеві алгоритми колоризації, включаючи моделі на основі GAN, Diffusion, Transformer та CNN;
    \item знайти та доповнити, за необхідності, датасет для навчання та тестування моделей колоризації;
    \item розробити власний алгоритм колоризації використовуючи нейромережеву модель;
    \item розробити програмну реалізацію системи автоматичної та інтерактивної колоризації з багатоваріантним генеруванням результатів;
    \item провести порівняння реалізованих методів за об'єктивними метриками та за суб'єктивним візуальним аналізом.
    % \item провести огляд опублікованих джерел за тематикою дослідження;
    % \item (наступний пункт, пов'язаний із теоретичним дослідженням);
    % \item (і ще один, наприклад, про експериментальну перевірку результатів);
    % \item (і взагалі, краще із науковим керівником проконсультуйтесь, як ваші завдання правильно писати).
\end{enumerate}

\emph{Об'єктом дослідження}
є процеси автоматичної та інтерактивної колоризації цифрових зображень і відео з багатоваріантним генеруванням результатів.
% є якісь процеси або явища загального 
% характеру (наприклад, <<інформаційні процеси в системах криптографічного 
% захисту>>).

\emph{Предметом дослідження}
є моделі, алгоритми та методи колоризації та генерації різних варіантів результату, а також метрики для порівняння якості кольоризації.
% є конкретний математичний чи фізичний 
% об'єкт, який розглядається у вашій роботі та який можна трактувати
% як певну властивість об'єкта дослідження (наприклад, <<моделі та методи
% диференціального криптоаналізу ітеративних симетричних блочних шифрів>>).

При розв'язанні поставлених завдань використовувались такі \emph{методи дослідження}:
методи глибинного навчання (GAN, Diffusion, Transformer та CNN моделі), методи обробки зображень, метрики технічної точності та інструменти оцінки перцепційної якості.
% і тут коротенький перелік (наприклад, але не обмежуючись: методи лінійної та абстрактної 
% алгебри, теорії імовірностей, математичної статистики, комбінаторного 
% аналізу, теорії кодування, теорії складності алгоритмів, методи 
% комп'ютерного та статистичного моделювання) 

\textbf{Наукова новизна.}
\begin{enumerate}
    \item інтеграція автоматичної та інтерактивної колоризації в єдиній системі з підтримкою багатоваріантного генерування результатів;
    \item свій власний алгоритм колоризації зображень і відео на основі нейромережевої моделі;
    \item порівняння саме цих різних алгоритмів колоризації для зображень і відео за об'єктивними та суб'єктивними критеріями;
\end{enumerate}
%  отриманих результатів полягає
% ... -- тут необхідно 
% перелічити, що саме нового з точки зору науки несе ваша робота. До усіх 
% тверджень, які сюди виносяться, подумки (а іноді й явним чином) потрібно 
% ставити слово <<вперше>> -- і ці твердження повинні залишатись істинними.

\textbf{Практичне значення.}
Створення програмної системи, яка дозволяє автоматично та інтерактивно колоризувати зображення й відео та отримувати кілька варіантів результату.
Розроблене рішення може бути використане у сфері цифрової реставрації матеріалів, у художніх (творчих) та дизайнерських (практично-функціональних) застосуваннях, а також як інструмент для дослідників у галузі комп'ютерного зору та візуальної аналітики (Наприклад якщо вони мають вже готовий графік і хочуть до нього додати або змінити колір).

% результатів полягає... -- тут необхідно 
% зазначити практичну користь від результатів вашої роботи. Що саме можна 
% покращити, підвищити (або знизити), зробити гарного (або уникнути 
% поганого) після вашого дослідження.

% \textbf{Апробація результатів та публікації.} Наприкінці вступу необхідно 
% зазначити перелік конференцій, семінарів та публікацій, в яких викладено 
% результати вашої роботи. Якщо результати вашої роботи ніде не 
% доповідались, опускайте даний абзац.