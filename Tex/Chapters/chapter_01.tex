%!TEX root = ../thesis.tex
\chapter{Теоретичні основи колоризації зображень та відео}
\label{chap:review}  %% відмічайте кожен розділ певною міткою -- на неї наприкінці необхідно посилатись

    У цьому розділі розглядаються основні теоретичні аспекти колоризації зображень та відео, включно з історією розвитку методів, класичними алгоритмічними підходами та сучасними нейромережевими моделями.
    Особлива увага приділяється математичному опису процесу колоризації, а також методам оцінки якості отриманих результатів.

    % На початку кожного розділу рекомендується вставити одне-два-абзац речень, 
    % у яких коротенько представили, про що тут взагалі буде мова.

\section{Поняття та історія колоризації}

    % \cite{2023-cold-diffusion}
    % \cite{2023-cold-diffusion-official}

    % Перший розділ повинен бути присвячений огляду попередніх результатів за 
    % тематикою вашого дослідження. У даному розділі повинні міститись вс' 
    % визначення та описи, необхідні для подальшого викладення матеріалу, та результати 
    % ваших попередників.

    % Зауважимо, що наводити детальні доведення не ваших результатів необхідно 
    % наводити лише тоді, коли вони містять якусь вкрай важливу інформацію для 
    % саме ваших результатів.

    % Також зауважимо, що абсолютно на всі не ваші результати повинні стояти 
    % належним чином оформлені посилання.

    % Розмір першого (оглядового) розділу не повинен перевищувати третини вашої 
    % дипломної роботи (без урахування додатків).


\section{Основні підходи до колоризації}
\section{Інтерактивна колоризація та багатоваріантні результати}
\section{Метрики якості колоризації}

\chapconclude{\ref{chap:review}}

    % Наприкінці кожного розділу ви повинні навести коротенькі підсумки по його 
    % результатах. Зокрема, для оглядового розділу в якості висновків необхідно 
    % зазначити, які задачі у даній тематиці вже були розв'язані, а саме 
    % поставлена вами задача розв'язана не була (або розв'язана погано), тому у 
    % наступних розділах ви її й розв'язуєте.

    % Якщо ваш звіт складається з одного розділу, пропускайте висновок до 
    % нього~-- він повністю включається в загальні висновки до роботи