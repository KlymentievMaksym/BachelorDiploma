%!TEX root = ../thesis.tex
\chapter{Теоретичні основи колоризації зображень та відео}
\label{chap:review}

    У цьому розділі розглядаються основні теоретичні аспекти колоризації зображень та відео, включно з історією розвитку методів, класичними алгоритмічними підходами та сучасними нейромережевими моделями.
    Особлива увага приділяється архітектурам від згорткових мереж до дифузійних моделей, а також методам оцінки якості отриманих результатів.
    % математичному опису процесу колоризації як некоректно поставленої задачі (ill-posed problem), еволюції 
\section{Поняття та історія колоризації}

    Колоризація — це процес додавання кольорової інформації до монохромних зображень або відео.
    З математичної точки зору, цю задачу можна сформулювати як відображення одноканального зображення (grayscale) у триканальне кольорове зображення (наприклад, у просторі RGB, CIELAB або YUV) \cite{2020-survey}. Оскільки одному значенню яскравості може відповідати безліч кольорових відтінків, задача є невизначеною та має "один до багатьох" (one-to-many) характер, що вимагає використання апріорних знань про семантику сцени.

    Історично методи колоризації можна поділити на три основні етапи:
    \begin{enumerate}
        \item \textbf{Ручна та напівавтоматична колоризація:} вимагала значних зусиль художників або використання простих евристик на основі сегментації.
        \item \textbf{Методи на основі глибокого навчання (CNN та GAN):} поява великих наборів даних (ImageNet, COCO) дозволила тренувати згорткові мережі для передбачення кольору \cite{2016-deep-learning-official}. Знаковим став підхід \textit{Colorful Image Colorization} \cite{2016-colorful-image-colorization}, де задачу було переформульовано з класичної регресії на класифікацію. Замість того, щоб мінімізувати усереднену похибку (що призводить до тьмяних кольорів), модель передбачає розподіл ймовірностей для кожного пікселя по палітрі відтінків.
        \item \textbf{Генеративні моделі (Transformers та Diffusion Models):} сучасний етап, що характеризується використанням механізмів уваги та імовірнісних моделей дифузії для генерації високоякісних текстур та забезпечення семантичної узгодженості \cite{2023-diffusion-models-survey, 2023-diffusion-models-survey-official}.  % 2023-diffusion-models-survey, 
    \end{enumerate}

    

    Розвиток методів глибокого навчання, зокрема поява архітектур U-Net та GAN (Generative Adversarial Networks), дозволив перейти від простого розфарбовування до генерації фотореалістичних зображень \cite{2017-image-to-image-translation, 2017-image-to-image-translation-official}. Останні дослідження все частіше фокусуються на дифузійних моделях, які демонструють state-of-the-art результати у задачах синтезу зображень \cite{2022-palette-image-to-image-diffusion-models, 2022-palette-image-to-image-diffusion-models-official}.

\section{Основні підходи до колоризації}

    Сучасні підходи до автоматичної колоризації базуються на різних архітектурах нейронних мереж. Розглянемо ключові з них.

    \subsection{Згорткові нейронні мережі (CNN)}
    Ранні підходи використовували автоенкодери для прямого передбачення каналів кольоровості (наприклад, $a$ та $b$ у просторі CIE Lab). Ласссон та ін. \cite{2016} запропонували використовувати ознаки, вивчені для класифікації об'єктів, для задачі колоризації, демонструючи тісний зв'язок між семантикою об'єкта та його кольором.
    Інший популярний метод, \textit{Deep Koalarization} \cite{2017-deep-koalarization}, використовує Inception-ResNet-v2 для вилучення ознак та подвійний декодер для відновлення кольору.

    \subsection{Генеративно-змагальні мережі (GAN)}
    Для подолання проблеми "сірості" (desaturation), властивої методам на основі функції втрат L2, були запропоновані GAN. Модель \textit{Pix2Pix} \cite{2017-image-to-image-translation} використовує умовний GAN (cGAN) для перетворення зображень, що дозволяє генерувати більш чіткі та реалістичні результати.
    Метод \textit{ChromaGAN} \cite{2020-chromagan} поєднує геометричну інформацію з семантичним розподілом класів для покращення правдоподібності розфарбовування. Також існують підходи для непарного навчання, такі як CycleGAN \cite{2017-unpaired-image-to-image-translation}, що дозволяють тренувати моделі без наявності пар "чорно-біле — кольорове".

    \subsection{Трансформери}
    З появою механізму уваги (Self-Attention) дослідники почали застосовувати трансформери для колоризації. Наприклад, \textit{Colorization Transformer} \cite{2021} та \textit{UniColor} \cite{2022-unicolor} використовують архітектуру трансформерів для моделювання довгих залежностей у зображенні, що покращує узгодженість кольорів на великих відстанях.

    

    \subsection{Дифузійні моделі}
    Найбільш перспективним напрямком на сьогодні є імовірнісні дифузійні моделі (Denoising Diffusion Probabilistic Models — DDPM). Вони працюють шляхом поступового видалення шуму з випадкового сигналу, керуючись вхідним чорно-білим зображенням.
    \begin{itemize}
        \item \textbf{Palette} \cite{2022-palette-image-to-image-diffusion-models} — універсальний фреймворк для перетворення зображень, який перевершує GAN та регресійні моделі без необхідності налаштування під конкретну задачу.
        \item \textbf{DDColor} \cite{2023-ddcolor} використовує подвійні декодери та запити кольору (color queries) для досягнення фотореалістичності.
        \item \textbf{Cold Diffusion} \cite{2023-cold-diffusion} пропонує підхід до інверсії довільних перетворень зображень без використання гаусового шуму, що відкриває нові можливості для відновлення зображень.
    \end{itemize}

    Для відео ключовою проблемою є часова узгодженість (temporal consistency). Сучасні методи, такі як \textit{Stable Video Diffusion} \cite{2023-stable-video-diffusion} та \textit{VanGogh} \cite{2025-vangogh}, адаптують дифузійні моделі для обробки послідовностей кадрів. Інші підходи, наприклад \cite{2019-deep-exemplar-based-video-colorization} та \cite{2024-temporally-consistent-video-colorization}, використовують поширення ознак (feature propagation) та саморегуляризацію для уникнення мерехтіння.

\section{Інтерактивна колоризація та багатоваріантні результати}

    Оскільки колоризація є суб'єктивною задачею, важливу роль відіграють методи, що дозволяють користувачеві впливати на результат.

    \textbf{Колоризація на основі прикладів (Exemplar-based):}
    Методи, такі як \textit{Deep Exemplar-based Colorization} \cite{2018-deep-exemplar-based-colorization}, використовують референсне кольорове зображення для перенесення колірної палітри на цільове чорно-біле. Це дозволяє досягти високої точності, якщо приклад підібрано вдало.

    \textbf{Колоризація на основі тексту (Language-based):}
    З розвитком мультимодальних моделей (CLIP) з'явилася можливість керувати кольором за допомогою текстових описів.
    \begin{itemize}
        \item \textbf{L-CAD} \cite{2023-l-cad} використовує дифузійні пріори для колоризації на основі описів будь-якого рівня деталізації.
        \item \textbf{Diffusing Colors} \cite{2023-diffusing-colors} пропонує фреймворк, де текстові підказки керують процесом дифузії, забезпечуючи баланс між автоматизацією та контролем.
    \end{itemize}

    \textbf{Колоризація на основі підказок користувача (User-guided):}
    Методи \textit{Real-Time User-Guided Image Colorization} \cite{2017-real-time-user-guided-image-colorization} та \textit{iColoriT} \cite{2023-icolorit} дозволяють користувачеві ставити кольорові точки або штрихи, які мережа поширює на відповідні семантичні регіони, використовуючи, наприклад, Vision Transformer для кращого розуміння контексту.

    Сучасна тенденція, як показано в \textit{Control Color} \cite{2025} та \textit{UniColor} \cite{2022-unicolor}, полягає у створенні уніфікованих фреймворків, що підтримують різні типи умов (текст, приклад, штрихи) в одній моделі.

    

\section{Метрики якості колоризації}

    Оцінка якості колоризації є складною через відсутність єдиного "правильного" розв'язку. Використовуються наступні групи метрик:

    \begin{enumerate}
        \item \textbf{Піксельні метрики:} PSNR (Peak Signal-to-Noise Ratio), SSIM (Structural Similarity Index), L1/L2 відстані. Вони порівнюють результат з оригінальним кольоровим зображенням (Ground Truth). Однак ці метрики часто не корелюють з людським сприйняттям якості, оскільки "інший" колір не обов'язково є "неправильним" \cite{2016-colorful-image-colorization}.
        \item \textbf{Перцептивні метрики:} LPIPS (Learned Perceptual Image Patch Similarity) та FID (Fréchet Inception Distance). FID, зокрема, широко використовується для оцінки дифузійних моделей \cite{2023-diffusion-models-survey}, оскільки оцінює відстань між розподілами ознак згенерованих та реальних зображень.
        \item \textbf{Семантичні оцінки та дослідження користувачів:} "Тест Тюрінга на колоризацію" (Colorization Turing Test), запропонований Zhang et al. \cite{2016-colorful-image-colorization}, де учасники намагаються відрізнити згенероване зображення від справжнього.
        \item \textbf{Оцінка барвистості (Colorfulness):} Використовується для перевірки того, чи не є результат тьмяним або надмірно насиченим (метрика C-score).
    \end{enumerate}

\chapconclude{\ref{chap:review}}

    У цьому розділі було проведено огляд теоретичних основ та сучасних методів колоризації зображень і відео. Було показано, що еволюція методів пройшла шлях від простих евристик до складних генеративних моделей, таких як GAN та дифузійні моделі (Palette, DDColor, Cold Diffusion).
    
    Аналіз літератури свідчить, що в той час як задача безумовної колоризації досягла значних успіхів у реалізмі, існують відкриті питання в області керованості процесом (controlability) та часової узгодженості відео (temporal consistency). Зокрема, мультимодальні підходи (текст + зображення) та уніфіковані дифузійні фреймворки є найбільш актуальним напрямком досліджень.
    
    У наступних розділах буде запропоновано власний підхід до вирішення задачі [Уточнити задачу, наприклад: інтерактивної колоризації відео на основі дифузійних моделей], що базується на проаналізованих архітектурах та спрямований на покращення [вказати мету, наприклад: стабільності кольорів у часі].
% \chapter{Теоретичні основи колоризації зображень та відео}
% \label{chap:review}  %% відмічайте кожен розділ певною міткою -- на неї наприкінці необхідно посилатись

%     У цьому розділі розглядаються основні теоретичні аспекти колоризації зображень та відео, включно з історією розвитку методів, класичними алгоритмічними підходами та сучасними нейромережевими моделями.
%     Особлива увага приділяється математичному опису процесу колоризації, а також методам оцінки якості отриманих результатів.

%     % На початку кожного розділу рекомендується вставити одне-два-абзац речень, 
%     % у яких коротенько представили, про що тут взагалі буде мова.

% \section{Поняття та історія колоризації}

%     % \cite{2023-cold-diffusion}
%     % \cite{2023-cold-diffusion-official}

%     % Перший розділ повинен бути присвячений огляду попередніх результатів за 
%     % тематикою вашого дослідження. У даному розділі повинні міститись вс' 
%     % визначення та описи, необхідні для подальшого викладення матеріалу, та результати 
%     % ваших попередників.

%     % Зауважимо, що наводити детальні доведення не ваших результатів необхідно 
%     % наводити лише тоді, коли вони містять якусь вкрай важливу інформацію для 
%     % саме ваших результатів.

%     % Також зауважимо, що абсолютно на всі не ваші результати повинні стояти 
%     % належним чином оформлені посилання.

%     % Розмір першого (оглядового) розділу не повинен перевищувати третини вашої 
%     % дипломної роботи (без урахування додатків).


% \section{Основні підходи до колоризації}
% \section{Інтерактивна колоризація та багатоваріантні результати}
% \section{Метрики якості колоризації}

% \chapconclude{\ref{chap:review}}

%     % Наприкінці кожного розділу ви повинні навести коротенькі підсумки по його 
%     % результатах. Зокрема, для оглядового розділу в якості висновків необхідно 
%     % зазначити, які задачі у даній тематиці вже були розв'язані, а саме 
%     % поставлена вами задача розв'язана не була (або розв'язана погано), тому у 
%     % наступних розділах ви її й розв'язуєте.

%     % Якщо ваш звіт складається з одного розділу, пропускайте висновок до 
%     % нього~-- він повністю включається в загальні висновки до роботи