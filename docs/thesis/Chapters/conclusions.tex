%!TEX root = ../thesis.tex
%!TEX root = ../thesis.tex
\conclusions

    У дипломній роботі вирішено актуальну науково-практичну задачу розробки інтерактивної системи колоризації зображень та відео, здатної генерувати фотореалістичні та варіативні результати. У ході виконання роботи були отримані наступні результати:

    1.  \textbf{Проведено глибокий аналіз сучасного стану проблеми колоризації.}
    Встановлено, що класичні методи оптимізації та ранні нейромережеві підходи (на базі CNN з функцією втрат L2) страждають від проблеми "усереднення" кольорів, що призводить до ненасичених, сірих зображень. Порівняльний аналіз показав, що найбільш перспективним напрямком є використання імовірнісних дифузійних моделей (Diffusion Models), які, на відміну від GAN, забезпечують вищу стабільність навчання та кращу деталізацію текстур, хоча і потребують більших обчислювальних ресурсів.

    2.  \textbf{Розроблено алгоритм та програмну реалізацію системи інтерактивної колоризації.}
    За основу взято архітектуру умовної дифузійної моделі з механізмом уваги (Attention). Ключовою особливістю розробленої системи є реалізація механізму багатоваріантної генерації, що дозволяє отримувати декілька правдоподібних варіантів забарвлення для одного вхідного зображення шляхом стохастичного семплювання. Також впроваджено гібридну систему керування процесом, яка приймає як глобальні текстові описи, так і локальні точкові підказки користувача.

    3.  \textbf{Виконано експериментальне дослідження ефективності системи.}
    Тестування на наборі даних COCO-Stuff підтвердило перевагу запропонованого методу над існуючими аналогами (зокрема, методами на основі GAN).
    \begin{itemize}
        \item Значення метрики FID (Fréchet Inception Distance) вдалося знизити до \textbf{24.2}, що на \textbf{25\%} краще за показники популярного методу DeOldify (32.5), що свідчить про вищу реалістичність згенерованих зображень.
        \item Суб'єктивна оцінка якості користувачами (Mean Opinion Score) склала \textbf{4.2} бала з 5 можливих, що значно перевищує оцінку повністю автоматичних методів (3.6).
        \item Доведено ефективність інтерактивного режиму: додавання всього 2-х точкових підказок дозволяє підвищити точність відновлення кольору (метрика PSNR) у складних сценах в середньому на \textbf{3.3~дБ}.
    \end{itemize}

    4.  \textbf{Визначено межі застосування розробленої системи.}
    Встановлено, що через специфіку ітеративного процесу дифузії середній час обробки одного зображення складає 4–5 секунд. Це робить систему ідеальною для задач професійної реставрації архівних матеріалів та художньої обробки фотографій, де пріоритетом є якість та контроль, але обмежує її використання в системах реального часу.

    \textbf{Напрямки подальших досліджень.}
    Основним вектором розвитку є оптимізація швидкодії алгоритму. Перспективним вбачається перехід до латентних дифузійних моделей (Latent Diffusion Models), що дозволить виконувати генерацію у стиснутому просторі ознак та значно пришвидшити процес. Також для покращення колоризації відео доцільно інтегрувати модулі часової уваги (Temporal Attention) або механізми оптичного потоку безпосередньо в архітектуру нейронної мережі для усунення ефекту мерехтіння без необхідності пост-обробки.
% % створюємо Висновки до всієї роботи
% \conclusions
% Загальні висновки до роботи повинні підсумовувати усі ваші досягнення у 
% даному напрямку досліджень.

% За кожним пунктом завдань, поставлених у вступі, у висновках повинен 
% міститись звіт про виконання: виконано, не виконано, виконано частково (І 
% чому саме так). Наприклад, якщо першим поставленим завданням у вас іде 
% <<огляд літератури за тематикою досліджень>>, то на початку висновків ви 
% повинні зазначити, що <<у ході даної роботи був проведений аналіз 
% опублікованих джерел за тематикою (...), який показав, що (...)>>. Окрім 
% простої констатації про виконання ви повинні навести, які саме результати 
% ви одержали та проінтерпретувати їх з точки зору поставленої задачі, мети 
% та загальної проблематики.

% В ідеалі загальні висновки повинні збиратись з висновків до кожного 
% розділу, але ідеал недосяжний. :) Однак висновки не повинні містити 
% формул, таблиць та рисунків. Дозволяється (та навіть вітається) 
% використовувати числа (на кшталт <<розроблена методика дозволяє підвищити 
% ефективність пустопорожньої балаканини на $2.71\%$>>).

% Наприкінці висновків необхідно зазначити напрямки подальших досліджень: 
% куди саме, як вам вважається, необхідно прямувати наступним дослідникам у 
% даній тематиці.
